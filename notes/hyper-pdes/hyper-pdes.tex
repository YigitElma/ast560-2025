\documentclass[12pt]{article}

\usepackage{tikz}
\usetikzlibrary{decorations.markings}

%\usepackage{geometry}
%\geometry{paperwidth=170mm, paperheight=240mm, left=42pt, top=40pt, textwidth=280pt, marginparsep=20pt, marginparwidth=100pt, textheight=560pt, footskip=40pt}

%\documentclass[12pt]{amsart}
\usepackage[utf8]{inputenc}
\usepackage{times}
\usepackage{geometry}
\usepackage{tabularx}
\usepackage{amsmath}
\usepackage{amsfonts}
\usepackage{amsthm}
\usepackage{amssymb}
\usepackage{stmaryrd}
\usepackage{color}
\usepackage{sidecap}
\usepackage{physics}
\usepackage{graphicx}
\usepackage{color}
\usepackage{verse}
\usepackage{hyperref}
\usepackage{tensor}
\usepackage{marginnote}
\usepackage{wrapfig}
\usepackage[font={small}]{caption}

\newcommand{\gke}{\texttt{Gkeyll}}

\newcommand{\attrib}[1]{
\nopagebreak{\raggedleft\footnotesize #1\par}\vskip0.1in}
\renewcommand{\poemtitlefont}{\normalfont\large\itshape\centering}

\newtheorem{proposition}{Proposition}
\newtheorem{theorem}{Theorem}
\newtheorem{lemma}{Lemma}
\newtheorem{remark}{Remark}
\newtheorem{definition}{Definition}
\newtheorem{principle}{Principle}
\theoremstyle{definition}
\newtheorem{exmp}{Example}

\theoremstyle{definition}
\newtheorem{entry}{Entry}

\theoremstyle{definition}
\newtheorem{exer}{Exercise}

% auto-scaled figured
\newcommand{\incfig}{\centering\includegraphics}
\setkeys{Gin}{width=0.9\linewidth,keepaspectratio}

% Commonly used macros
\newcommand{\eqr}[1]{Eq.\thinspace(#1)}
\newcommand{\pfrac}[2]{\frac{\partial #1}{\partial #2}}
\newcommand{\pfracc}[2]{\frac{\partial^2 #1}{\partial #2^2}}
\newcommand{\pfraccc}[1]{\frac{\partial^2 }{\partial #1^2}}
\newcommand{\pfraca}[1]{\frac{\partial}{\partial #1}}
\newcommand{\pfracb}[2]{\partial #1/\partial #2}
\newcommand{\pfracbb}[2]{\partial^2 #1/\partial #2^2}
\newcommand{\spfrac}[2]{{\partial_{#1}} {#2}}
\newcommand{\mvec}[1]{\mathbf{#1}}
\newcommand{\bmvec}[1]{\breve{\mathbf{#1}}}
\newcommand{\gvec}[1]{\boldsymbol{#1}}
\newcommand{\script}[1]{\mathpzc{#1}}
\newcommand{\gcn}{\nabla}
\newcommand{\gcs}{\nabla_{\mvec{x}}}
\newcommand{\gvs}{\nabla_{\mvec{v}}}

\newcommand{\resetall}{\setcounter{entry}{0}\setcounter{equation}{0}\setcounter{footnote}{0}}

\newcommand{\cbas}[1]{\gvec{\sigma}_{#1}}
\newcommand{\xbas}{\gvec{\sigma}_{1}}
\newcommand{\ybas}{\gvec{\sigma}_{2}}
\newcommand{\zbas}{\gvec{\sigma}_{3}}

\newcommand{\basis}[1]{\mvec{e}_{#1}}
\newcommand{\bbasis}[1]{\breve{\mvec{e}}_{#1}}
\newcommand{\dbasis}[1]{\mvec{e}^{#1}}
\newcommand{\bdbasis}[1]{\breve{\mvec{e}}^{#1}}

\newcommand{\nbasis}[1]{\hat{\mvec{e}}_{#1}}
\newcommand{\ndbasis}[1]{\hat{\mvec{e}}^{#1}}

\newcommand{\gbasis}[2]{\mvec{#1}_{#2}}
\newcommand{\gdbasis}[2]{\mvec{#1}^{#2}}

\newcommand{\veps}{\gvec{\varepsilon}}
\newcommand{\bdum}{\breve{\_}}
\newcommand{\vecspace}{\mathcal{V}}
\newcommand{\tvecspace}{T_pM}
\newcommand{\vnorm}[1]{\lVert{#1}\rVert}

\newcommand{\gsel}[2]{\langle{#1}\rangle_{#2}}
\newcommand{\gzsel}[1]{\langle {#1} \rangle}

\newcommand{\uln}[1]{\underline{#1}}

\newcommand{\dprod}{\mathfrak{D}}

\newcommand{\dvol}{\thinspace d^3\mvec{x}}
\newcommand{\dsurf}{\thinspace ds}

% Make the items smaller
\newcommand{\cramplist}{
	\setlength{\itemsep}{0in}
	\setlength{\partopsep}{0in}
	\setlength{\topsep}{0in}}
\newcommand{\cramp}{\setlength{\parskip}{.5\parskip}}
\newcommand{\zapspace}{\topsep=0pt\partopsep=0pt\itemsep=0pt\parskip=0pt}

%\marginpar{\raggedleft\footnotesize Margin note}

\title{AST560 2025: Hyperbolic Partial Differential Equations}%
\author{Ammar H. Hakim}%
\date{\today}

\begin{document}
\maketitle

\tableofcontents

\section{On Propagation of Small Perturbations. Definition of
  Hyperbolic PDEs.}

Hyperbolic partial differential equation (PDEs) appear everywhere in
physics. A simple reason for this is that time exists (as far as we
know), and that there is a maximum speed at which causal signals can
propagate. In this section we first work through an informal
definition of hyperbolic PDEs, giving the rigorous definition later.

\begin{definition}[Hyperbolic PDE (Informal)]
  A hyperbolic PDE is in one in which small perturbations around a
  uniform equilibrium propagate at a finite speed without any damping
  or dispersion.
\end{definition}

To understand how this definition leads to a mathematical condition,
let us look at a generic system of first-order partial differential
equations in 1D
\begin{align}
  \pfrac{\mvec{Q}}{t} + \pfrac{\mvec{F}}{x} = 0
  \label{eq:hyp-1d}
\end{align}
where $\mvec{Q}(x,t)$ is a column vector of $m$ \emph{conserved
  variables} and $\mvec{F} = \mvec{F}(\mvec{Q})$ is the corresponding
column vector of \emph{fluxes}. To understand how a small perturbation
propagates in this system we will linearize around a uniform initial
condition $\mvec{Q}(x,0) = \mvec{Q}_0$ by writing
\begin{align}
  \mvec{Q}(x,t) = \mvec{Q}_0 + \mvec{Q}_1(x,t)
\end{align}
where $\mvec{Q}_1(x,t)$ is a small perturbation. Substituting this in
\eqr{\ref{eq:hyp-1d}} we get
\begin{align}
  \pfrac{\mvec{Q}_1}{t} + 
  \frac{\partial}{\partial x} \mvec{F}(\mvec{Q}_0 + \mvec{Q}_1)
  =
  0.
\end{align}
Expanding
\begin{align}
  \mvec{F}(\mvec{Q}_0 + \mvec{Q}_1)
  =
  \mvec{F}(\mvec{Q}_0) + \pfrac{\mvec{F}}{\mvec{Q}}(\mvec{Q}_0)
  \mvec{Q}_1
  + \ldots
\end{align}
we get
\begin{align}
  \pfrac{\mvec{Q}_1}{t} + 
  \mvec{A}(\mvec{Q}_0)
  \pfrac{\mvec{Q}_1}{x}
  =
  0.
  \label{eq:lin-hyp-1d}
\end{align}
Here we have dropped all quadratic and higher nonlinearities and
defined
\begin{align}
  \mvec{A} \equiv \pfrac{\mvec{F}}{\mvec{Q}}
\end{align}
as the \emph{flux Jacobian}. Note that the flux Jacobian is a
$m\times m$ matrix and, in general, will depend on the state
$\mvec{Q}$. When $\mvec{A}$ does not depend on $\mvec{Q}$ the system
is called \emph{linear}. The system \eqr{\ref{eq:lin-hyp-1d}} is the
linearized equation for the propagation of the perturbation
$\mvec{Q}_1$. To understand the structure of this equation we can look
at a single mode $\mvec{Q}_1(x,t) = e^{ikx} e^{-i\omega t}$ to get
\begin{align}
  [k \mvec{A}(\mvec{Q}_0) -  \omega\mvec{I}] \mvec{Q}_1 = 0
\end{align}
where $\mvec{I}$ is a $m \times m$ unit matrix. This is an eigenvalue
problem and we see that the propagation of the mode will depend on the
eigenvalues of the flux Jacobian $\mvec{A}(\mvec{Q}_0)$. Let
$\lambda^p$ be the eigenvalues of $\mvec{A}(\mvec{Q}_0)$. Then the
\emph{dispersion relation} is
\begin{align}
  \omega^p = k \lambda^p.
\end{align}
Clearly, the mode will propagate without any damping if all the
$\lambda^p$ are \emph{real}. Also, there is no dispersion in this
system as both the phase- and group-velocities are constant. Hence,
this linearization shows that the PDE \eqr{\ref{eq:hyp-1d}} will be
hyperbolic if the eigenvalues of the flux Jacobian are \emph{all
  real}.

Though we arrived at the condition for hyperbolicity via a
linearization, this condition is, in fact, the defining property of
hyperbolic PDEs even in the nonlinear case. This leads to the
rigorous definition:

\begin{definition}[Hyperbolic PDE]
  A PDE
  \begin{align*}
    \pfrac{\mvec{Q}}{t} + \pfrac{\mvec{F}}{x} = 0
  \end{align*}
  is called \emph{hyperbolic} if the flux Jacobian
  \begin{align*}
    \mvec{A} \equiv \pfrac{\mvec{F}}{\mvec{Q}}
  \end{align*}
  has all real eigenvalues and a complete set of right eigenvectors
  (or, in other words, is diagonalizable) for all valid solution
  states $\mvec{Q}$. The system is called \emph{strictly hyperbolic}
  if all eigenvalues are distinct and non-zero.
\end{definition}

\begin{remark}
  Note that in this definition we have added an additional condition:
  that $\mvec{A}$ must be diagonalizable. We will see why this
  condition is needed below. When this condition is relaxed strange
  things can happen, for example, the formation of delta-function
  solutions. Note we also added the phrase ``for all valid solution
  states''. In general, not all solution states are valid. For
  example, for the Euler equations, which are hyperbolic, valid states
  are those in which the \emph{density and pressure are
    positive}. Often the valid states of a PDE are also referred to as
  their \emph{invariant domains}.
\end{remark}

\begin{remark}
  Most of the hyperbolic PDEs we encounter in physics are \emph{not}
  strictly hyperbolic. There are often repeated eigenvalues and even
  eigenvalues that are zero.
\end{remark}

We will denote $\lambda^p$ as the eigenvalues of the flux Jacobian,
and $\mvec{r}^p$, $p=1,\ldots,m$ as the complete set of right
eigenvectors (column vectors). Corresponding to these right
eigenvectors we will denote the left eigenvectors (row vectors) as
$\mvec{l}^p$. Note we must have
$\mvec{l}^p \mvec{r}_q = \delta\indices{^p_q}$.

We can extend this definition to a PDE in multiple dimensions as
follows.
\begin{definition}[Multi-Dimensional Hyperbolic PDE]
  A PDE
  \begin{align*}
    \pfrac{\mvec{Q}}{t} 
    + \pfrac{\mvec{F}}{x}
    + \pfrac{\mvec{G}}{y}
    + \pfrac{\mvec{H}}{z} = 0
  \end{align*}
  is called \emph{hyperbolic} if, for all unit vectors
  $\mvec{n} = n_x \cbas{x} + n_y \cbas{y} + n_z \cbas{z}$, the flux
  Jacobian
  \begin{align*}
    \mvec{A} \equiv 
    n_x \pfrac{\mvec{F}}{\mvec{Q}}
    + n_y \pfrac{\mvec{G}}{\mvec{Q}}
    + n_y \pfrac{\mvec{H}}{\mvec{Q}}
  \end{align*}
  has all real eigenvalues and a complete set of right eigenvectors
  (or, in other words, is diagonalizable) for all valid solution
  states $\mvec{Q}$. We call the hyperbolic PDE \emph{isotropic} if
  the eigenvalues of $\mvec{A}$ do not depend on the unit vector
  $\mvec{n}$.
\end{definition}

Physically, if a hyperbolic PDE is isotropic, then the speed of
propagation does not depend on the direction of the wave. Most
hyperbolic PDEs one encounters are isotropic.

Though we have written the PDE (see \eqr{\ref{eq:hyp-1d}}) in
conservation form, a more general form is the \emph{quasilinear} form
\begin{align}
  \pfrac{\mvec{V}}{t} + \mvec{B}\pfrac{\mvec{V}}{x} = 0.
  \label{eq:ql-hyp-1d}
\end{align}
Here $\mvec{V}$ is a column vector of $m$ quantities and
$\mvec{B} = \mvec{B}(\mvec{V})$ is a $m\times m$ matrix that, in
general, depends on $\mvec{V}$. We call a quasilinear PDE hyperbolic
if the eigenvalues of $\mvec{B}$ are all real and $\mvec{B}$ is
diagonalizable. 

We can transform a equation written in conservation law form to the
quasilinear form, but it is not always possible to do the
opposite. Consider the 1D conservation law
\begin{align}
  \pfrac{\mvec{Q}}{t} + \pfrac{\mvec{F}}{x} = 0
\end{align}
and let $\mvec{V}$ be related by the invertible transform.
\begin{align}
  \mvec{Q} = \varphi(\mvec{V}).
\end{align}
Then, substituting this in the conservation law we get the quasilinear
form
\begin{align}
  \pfrac{\mvec{V}}{t}
  +
  \big(\varphi' \big)^{-1} \mvec{A} \varphi' \pfrac{\mvec{V}}{x} = 0.
\end{align}
Here $\varphi'$ is the Jacobian of the transform and $\mvec{A}$ is the
flux Jacobian of the conservation law. Define
\begin{align}
  \mvec{B} \equiv \big(\varphi' \big)^{-1} \mvec{A} \varphi'.
\end{align}
Let $\lambda^p$, $\mvec{r}^p$ and $\mvec{l}^p$ be the eigenvalues,
right- and left-eigenvectors of $\mvec{B}$ respectively. Then, we have
\begin{align}
  \mvec{B}\mvec{r}^p 
  = \lambda^p \mvec{r}^p
  = \big(\varphi' \big)^{-1} \mvec{A} \varphi' \mvec{r}^p.
\end{align}
Left multiply by $\varphi'$ to see that $\lambda^p$ are also the
eigenvalues of $\mvec{A}$, and $\varphi' \mvec{r}^p$ are the right
eigenvectors of $\mvec{A}$. In the same way we see that
$\mvec{l}^p \big(\varphi' \big)^{-1}$ are the left eigenvectors of
$\mvec{A}$. Hence, knowing the eigensytem of the quasilinear form of
the equations we can easily compute the eigensytem of the conservation
law form. Often, for complex nonlinear equations it is, in fact, much
easier to work with the quasilinear form and so the above procedure
is very useful.

\section{Examples of Hyperbolic PDEs}

We will now look at a sequence of examples of hyperbolic PDEs that
arize in fluid mechanics and plasma physics. One can multiple the
examples given below to many more systems: hyperbolic PDEs are
essentially universal, and arise 

\subsection{Linear Advection Equation}

One example we have already seen and studied earlier is the advection
equation
\begin{align}
  \pfrac{f}{t} + \pfraca{x}(u f) = 0.
\end{align}
This is the simplest \emph{linear} scalar hyperbolic PDE. It has a
single eigenvalue $\lambda^1 = u$. Despite its extreme simplicity, the
linear advection equation is the fundamental equation that provides
insight into solving more complex systems of equations, including
nonlinear equations, specially numerically.

\subsection{Burgers Equation}

The simplest nonlinear scalar hyperbolic PDE is the \emph{Burgers
  equation}
\begin{align}
  \pfrac{u}{t} + \pfraca{x} \left(\frac{1}{2} u^2 \right) = 0.
\end{align}
For this equation, $Q = u$ and $F(Q) = u^2/2$. Hence, the flux
Jacobian is just a scalar $A = u$, and the single eigenvalue is
$\lambda^1 = u$. Notice something very important: the eigenvalue
depends on the solution itself. Hence, the propagation speed, in
general, will vary from point to point, and, in fact, may change
signs.

\subsection{Maxwell Equations}

Maxwell equations are the simplest \emph{system} of linear hyperbolic
PDEs that occur in physics. In free-space, we can write them in curl
form as
\begin{align}
  \frac{\partial \mathbf{B}}{\partial t} + \nabla\times\mathbf{E} &= 0 \\
  \epsilon_0\mu_0\frac{\partial \mathbf{E}}{\partial t} -
  \nabla\times\mathbf{B} &= 0
\end{align}
Here, $\mathbf{E}$ is the electric field, $\mathbf{B}$ is the magnetic
flux density, and $\epsilon_0$, $\mu_0$ are permittivity and
permeability of free space. The speed of light is determined from
$c=1/(\mu_0\epsilon_0)^{1/2}$.

These are linear equations and hence the eigensytem is independent of
the value of the electromagnetic fields. In 1D Maxwell equations can
be written as
\begin{align}
  \frac{\partial }{\partial t}
  \left[
    \begin{matrix}
      E_x \\
      E_y \\
      E_z \\
      B_x \\
      B_y \\
      B_z
    \end{matrix}
  \right]
  +
  \frac{\partial }{\partial x}
  \left[
    \begin{matrix}
      0 \\
      c^2B_z \\
      -c^2B_y \\
      0 \\
      -E_z \\
      E_y
    \end{matrix}
  \right]
  =
  0.
\end{align}

The eigenvalues of this system are $-c,-c,c,c,0,0$. Note the repeated
eigenvalues and also the zero eigenvalues. Hence, Maxwell equations
are \emph{not} strictly hyperbolic. Despite this, they posses a
complete set of right eigenvectors. A simple calculation (easily done
in a computer algebra system) shows that the right eigenvectors are
the columns of the following matrix
\begin{align}
  \mvec{R}
  =
  \left[
    \begin{matrix}
    0&0&0&0&1&0 \\ 
    1&0&1&0&0&0 \\ 
    0&1&0&1&0&0 \\ 
    0&0&0&0&0&1 \\ 
    0&{{1}\over{c}}&0&-{{1}\over{c}}&0&0 \\ 
    -{{1}\over{c}}&0&{{1}\over{c}}&0&0&0
    \end{matrix}
  \right]  
\end{align}
and the left eigenvectors are the rows of the matrix
\begin{align}
  \mvec{L}
  =
  \left[
    \begin{matrix}
    0&{{1}\over{2}}&0&0&0&-{{c}\over{2}} \\
    0&0&{{1}\over{2}}&0&{{c}\over{2}}&0 \\
    0&{{1}\over{2}}&0&0&0&{{c}\over{2}} \\ 
    0&0&{{1}\over{2}}&0&-{{c}\over{2}}&0 \\
    1&0&0&0&0&0 \\ 
    0&0&0&1&0&0
    \end{matrix}
  \right].  
\end{align}

\subsection{The Isothermal Euler Equations}

The simplest system of \emph{nonlinear} hyperbolic PDEs that appears
in fluid mechanics is the isothermal Euler equations. These equations
describe the motion of an ideal fluid (that is, one with no viscosity
or heat-conduction) with a \emph{constant} temperature. In 1D these
equations can be written as
\begin{align}
  \frac{\partial}{\partial{t}}
  \left[
    \begin{matrix}
      \rho \\
      \rho u
    \end{matrix}
  \right]
  +
  \frac{\partial}{\partial{x}}
  \left[
    \begin{matrix}
      \rho u \\
      \rho u^2 + \rho a^2
    \end{matrix}
  \right]
  =
  0  
\end{align}
where $\rho$ is the fluid mass density, $u$ is the fluid velocity and
$a$ is a constant with units of speed. (This is, in fact, the sound
speed). The first of these equations is the \emph{continuity equation}
and the second is the \emph{momentum equation}.

To compute the eigenvalues of the isothermal Euler equation (and other
nonlinear systems) it is actually much easier to work with the
quasilinear form. The momentum equation can be written as
\begin{align}
  \rho\pfrac{u}{t} + 
  \underbrace{
  u \pfrac{\rho}{t}
  + u \pfraca{x}(\rho u)
  }_{0}
  + \rho u \pfrac{u}{x}
  +
  a^2 \pfrac{\rho}{x}
  =
  0.
\end{align}
where we used the continuity equation to eliminate the second and
third terms. Hence, we get that the isothermal Euler equation is
equivalent to the quasilinear system
\begin{align}
  \frac{\partial}{\partial{t}}
  \left[
    \begin{matrix}
      \rho \\
      u
    \end{matrix}
  \right]
  +
  \left[
    \begin{matrix}
      u & \rho \\
      a^2/\rho & u
    \end{matrix}
  \right]
  \frac{\partial}{\partial{x}}
  \left[
    \begin{matrix}
      \rho \\
       u
    \end{matrix}
  \right]
  =
  0.
\end{align}
Thus, we have
\begin{align}
  \mvec{B}
  =
  \left[
    \begin{matrix}
      u & \rho \\
      a^2/\rho & u
    \end{matrix}
  \right].  
\end{align}
Computing the eigenvalues of this matrix we get that the eigenvalues
of the isothermal Euler equations are $\lambda^{1,2} = u \pm a$. We
can also show that as long as $a > 0$, we have a complete set of
eigenvectors, hence showing that the isothermal Euler equations are
hyperbolic.

\subsection{The Euler Equations}

Probably the most important hyperbolic PDEs are the Euler equations
for ideal fluids. These equations are extremely useful, both by
themselves, and also as building blocks of more complex equations in
fluid dynamics and plasma physics. For example, the compressible
Navier-Stokes equations can be built from these by adding non-ideal
terms (viscosity and heat-conduction). For plasma physics, Euler
equations can describe \emph{two-fluid} plasmas, in which each plasma
species is treated as a separate charged fluid, coupled by
electromagnetic fields and collisions. In fact, these two-fluid
systems (with additional physics to include non-ideal terms) contain
physics beyond MHD and are required to properly study, for example,
magnetic reconnection, instabilities in various fusion systems, and
space and astrophysical plasmas.

In 1D the Euler equations can be written in conservative form as
\begin{align}
  \frac{\partial}{\partial{t}}
  \left[
    \begin{matrix}
      \rho \\
      \rho u \\
      \rho v \\
      \rho w \\
      E
    \end{matrix}
  \right]
  +
  \frac{\partial}{\partial{x}}
  \left[
    \begin{matrix}
      \rho u \\
      \rho u^2 + p \\
      \rho uv \\
      \rho uw \\
      (E+p)u
    \end{matrix}
  \right]
  =
  0
\end{align}
where
\begin{align}
  E = \rho \varepsilon + \frac{1}{2}\rho (u^2 + v^2 + w^2)
\end{align}
is the total energy and $\varepsilon$ is the internal energy of the
fluid. The pressure is given by an equation of state (EOS)
$p=p(\varepsilon, \rho)$. For an ideal gas the EOS is
\begin{align}
  p =(\gamma-1)\rho \varepsilon.
\end{align}
For a neutral fluid, usually we set $\gamma = 1.4$, while for a plasma
$\gamma = 5/3$.

Computing the eigensystem of the Euler equations is not
trivial. However, the task is made much more tractable by transforming
the system to quasilinear form and using a computer algebra system. We
find is that the eigenvalues are
\begin{align}
  \lambda^1, \lambda^2, \lambda^3, \lambda^4, \lambda^5
  =
  u-c, u, u, u, u+c,
\end{align}
where $c$ is the sound-speed given by
\begin{align}
  c &= \sqrt{\frac{\partial p}{\partial \rho} 
    + \frac{p}{\rho^2}\frac{\partial p}{\partial \varepsilon}}.  
\end{align}
In general, to compute the sound speed we need the equation of state
(EOS). Sometimes, the EOS is given as an analytical formula, but more
often is in tabulated form. The National Ignition Facility (NIF), for
example, is an experiment to collect the equation of state for
elements (and mixtures) used in nuclear weapons. Less aggressively
(but more energetically), for example, neutron star crust simulations
also require a semi-analytical EOS to properly simulate their
dynamics. The eigenvalues are all real if the sound-speed is
real. This, of course, depends on the EOS. For the ideal gas EOS, we
see that
\begin{align}
  c &= \sqrt{\frac{\gamma p}{\rho}}.
\end{align}
This remains real as long as $\rho > 0$ and $p > 0$. These conditions
on density and pressure (the invariant domain) restrict the valid
states a fluid can take.

The right eigenvectors are given by the columns of the matrix
\begin{align}
  R
  =
  \left[
    \begin{matrix}
      1 & 0 & 0 & 1 & 1 \\
      u-c & 0 & 0 & u & u+c \\
      v & 1 & 0 & v & v \\
      w & 0 & 1 & w & w \\
      h-uc & v & w & h-c^2/b & h+uc
    \end{matrix}
  \right].
  \label{eq:euler-rev}
\end{align}
Here, $h$ is the \emph{enthalpy} of the fluid given by
\begin{align}
  h &= (E+p)/\rho.
\end{align}
Also
\begin{align}
  b = \frac{1}{\rho}\frac{\partial p}{\partial \varepsilon}.
\end{align}

From this discussion it becomes clear that the Euler equations are an
extremely complex, nonlinear system of equations. Things become even
more complicated for general EOS. Of course, further extensions to
these fundamental systems are possible, in particular, to (special and
general) relativistic problems. For high-energy astrophysics problems,
in fact, one must use these relativistic extensions to the Euler
equations as the energies and speeds encountered in extreme plasma
environments (for example, around black holes and neutron stars) can
be ultra-relativistic.

\subsection{The Ideal MHD Equations}

In plasma physics the single most important set of equations are the
ideal MHD equations. These are the foundation of all of plasma
physics, including the theory of equilibrium and stability of tokamaks
and stellarators, and the proper understanding of everything from
space to astrophysical plasmas.

Ideal MHD is a pre-Maxwell theory in the sense that the displacement
currents are ignored (hence, no electromagnetic waves) and the plasma
is treated as a conducting fluid. The equations consist of the
\emph{continuity} equation
\begin{align}
  \pfrac{\rho}{t} + \nabla\cdot(\rho\mvec{u}) = 0,
\end{align}
and the \emph{momentum equation}
\begin{align}
  \frac{\partial \mathbf{u}}{\partial t}+\mathbf{u} \cdot \nabla \mathbf{u}
  +\frac{\nabla p}{\rho} &= 
                           {\frac{1}{\mu_{0}\rho}(\nabla \times \mathbf{B}) \times \mathbf{B}}
                           \label{eq:mhd-mom}  
\end{align}
where $\mvec{B}$ is the magnetic field. The evolution of the field is
determined by the \emph{induction equation}
\begin{align}
  \pfrac{\mvec{B}}{t} + \nabla\times\mvec{E} = 0
\end{align}
where $\mvec{E} = -\mvec{u}\times\mvec{B}$ (ideal Ohm's Law). Finally,
for an ideal plasma, the pressure evolves according to
\begin{align}
  \pfrac{p}{t} + \mvec{u}\cdot\nabla p = -\gamma p
  \nabla\cdot\mvec{u}.
  \label{eq:mhd-press}
\end{align}

Using some algebraic manipulations we can show that the momentum
equation can be written in the conservative form as
\begin{align}
  \frac{\partial}{\partial t}(\rho\mvec{u})
  + \nabla\cdot \mvec{T} = 0
  \label{eq:mhd-mom-cons}
\end{align}
where
\begin{align}
  \mvec{T} = \rho\mvec{u}\otimes\mvec{u}
  +
  \left(p + \frac{\mvec{B}^2}{2\mu_0} \right)\mvec{g}
  -\frac{1}{\mu_0} \mvec{B}\otimes\mvec{B}.
\end{align}
is a \emph{symmetric} second-order tensor describing the momentum
flux, and $\mvec{g}$ is the metric tensor.

Some rather hairy manipulations allow us to derive evolution equations
for the kinetic-energy
\begin{align}
  \frac{\partial}{\partial t}
  \left(
  \frac{1}{2}\rho\mvec{u}^2
  \right)
  +
  \nabla\cdot
  \left[
  \frac{1}{2}\rho\mvec{u}^2\mvec{u}
  +
  \left(p+\frac{\mvec{B}^2}{2\mu_0}\right)\mvec{u}
  \right]
  =
  \left(p+\frac{\mvec{B}^2}{2\mu_0}\right)\nabla\cdot\mvec{u}
  +
  \frac{1}{\mu_0}
  \mvec{u}\cdot
  \left[
  (\mvec{B}\cdot\nabla)\mvec{B}
  \right],
\end{align}
the internal energy 
\begin{align}
  \frac{\partial}{\partial t}\left(\frac{p}{\gamma-1}\right)
  +
  \nabla\cdot
  \left(\mvec{u}\frac{p}{\gamma-1}\right)
  =
  -p\nabla\cdot\mvec{u},
\end{align}
and the magnetic field energy
\begin{align}
  \frac{\partial}{\partial t}\left(\frac{\mvec{B}^2}{2\mu_0} \right)
  -\nabla\cdot
  \left(
  \frac{1}{\mu_0}(\mvec{u}\cdot\mvec{B})\mvec{B}
  -
  \frac{\mvec{B}^2}{2\mu_0}\mvec{u}
  \right)
  =
  -\frac{\mvec{B}^2}{2\mu_0}\nabla\cdot\mvec{u}
  -
  \frac{1}{\mu_0}\mvec{u}\cdot[(\mvec{B}\cdot\nabla)\mvec{B}].
\end{align}

These evolution equations for the various components of the energy
show how energy can be exchanged between various components of the
fluid and the field: internal energy and kinetic energies are
exchanged via the fluid compressibility, and kinetic and field
energies through a complicated set of terms involving compressibility
and currents and electric field (effectively, an
$\mvec{E}\cdot\mvec{J}$ term).

Rather neatly, if we add all the components of the energy up the
right-hand sides all cancel, and we get that the \emph{total energy}
\begin{align}
  \mathcal{E}
  \equiv
  \frac{1}{2}\rho\mvec{u}^2
  +
  \frac{p}{\gamma-1}
  +
  \frac{\mvec{B}^2}{2\mu_0}
\end{align}
evolves as
\begin{align}
  \pfrac{\mathcal{E}}{t}
  +
  \nabla\cdot
  \left[
  (\mathcal{E}+p^*)\mvec{u} - \frac{1}{\mu_0}(\mvec{u}\cdot\mvec{B})\mvec{B}
  \right]
  =
  0
\end{align}
where the total (fluid + magnetic-field) pressure is defined as
\begin{align}
  p^* \equiv p + \frac{\mvec{B}^2}{2\mu_0}.
\end{align}

Of course, the evolution of the total energy equation is very
important. However, I should point out that equally important are the
evolution of the components of the energy, as these fundamentally
determine energizaton processes, and also how turbulence manifests
itself in MHD. In fact, it is \emph{extremely difficult} (perhaps
impossible) to develop generic numerical schemes that account for both
proper transfer between various components of the energy, and also at
the same time account for shocks and other features of
compressibility. As nature would have it, these are the cutting-edge
regimes of interest for modern, high-energy plasma physics.

It is clear that computing the eigensystem of this formidable set of
equation is highly non-trivial. However, again, it is easiest to work
in the quasilinear form of the equations and use a computer algebra
system. One finds that indeed the ideal MHD equation is hyperbolic.

\section{Discontinous Solutions and Entropy Conditions}

One of the defining features of the hyperbolic PDEs is the formation
of \emph{shocks}, \emph{rarefaction fans} and \emph{contact
  discontinuities}. We will now see how these arise by looking first
at the simplest nonlinear scalar hyperbolic PDEs, the Burgers
equation. However, first consider the scalar quasilinear PDE
\begin{align}
  \pfrac{V}{t} + a(V)\pfrac{{V}}{x} = 0
\end{align}
for $-\infty < x < \infty$. For this system we simply have a single
eigenvalue, $\lambda^1 = a(V)$. This equation also indicates that if
we define \emph{characteristics} in the $(x,t)$ plane:
\begin{align}
  x = \lambda^1 t + C = a(V) t + C
\end{align}
where $C$ is some constant, then the solution will remain constant
along those characteristics. (Of course, different characteristics
will, in general, have different values of $V$).

\begin{figure}
  \setkeys{Gin}{width=0.95\linewidth,keepaspectratio}
  \incfig{burgers-shock.pdf} 
  \caption{To avoid multi-valued solutions due to converging
    characteristics (right) we must allow \emph{discontinuous}
    solutions to Burgers equation. These \emph{shock solutions} (left)
    are only defined \emph{weakly}, in the sense that they satisfy the
    weak-form of the equation and not the strong form.}
  \label{fig:burgers-shock}
\end{figure}

We can use this \emph{method of characteristics} to solve the Burgers
equation
\begin{align}
  \pfrac{u}{t} + \pfraca{x} \left(\frac{1}{2} u^2 \right) = 0
\end{align}
with initial conditions $u(x,0) = u_0(x)$. Here the characteristic
speed is just $u$, and hence the characteristics are $x = u t +
C$. Now consider some point $x_0$ at $t=0$. Then we must have
$C = x_0$. As the characteristic velocity at that point is $u_0(x_0)$,
the characteristic must be
\begin{align}
  x  = u_0(x_0) t + x_0.
  \label{eq:burg-char}
\end{align}
As the solution remains constant along characteristics, the exact
solution to Burgers equation can be easily written as
\begin{align}
  u(x,t) = u_0(x_0)
\end{align}
where $x_0$ is determined from \eqr{\ref{eq:burg-char}}. At this point
we may think we have found a general solution to the 1D Burgers
equation. However, it is not so. Consider the case shown in the
Fig.\thinspace\ref{fig:burgers-shock}. Here we see that
characteristics \emph{intersect} after some finite-time. As the
solutions are constant along characteristics this would mean that the
solution at the point of intersection are multi-valued! To avoid this
undesirable situation we must introduce \emph{shock solutions}, that
is, solutions that are \emph{discontinuous}. 

Discontinuous solutions, of course, do not have well-defined
deriviatives (their derivatives are \emph{distributions}, or delta
functions). Hence, we are led to consider the notion of
\emph{weak-solutions} of hyperbolic PDEs. These are ``derived'' by
multiplying \eqr{\ref{eq:hyp-1d}} by a compact, smooth function
$\phi(x,t)$ (that, a function that has smooth derivatives and vanishes
outside some bounded region) and integrating over all time and space:
\begin{align}
    \int_0^\infty  \int_{-\infty}^\infty \phi(x,t)
    \bigg[\pfrac{\mvec{Q}}{t} + \pfrac{\mvec{F}}{x}\bigg]\thinspace
    dx\thinspace dt = 0.
\end{align}
Integrating by parts we get
\begin{align}
    \int_0^\infty  \int_{-\infty}^\infty 
    \bigg[\pfrac{\phi}{t} \mvec{Q} + \pfrac{\phi}{x} \mvec{F}\bigg]\thinspace
    dx\thinspace dt
    =
    -
    \int_{-\infty}^\infty \phi(x,0) \mvec{Q}(x,0) dx.  
  \label{eq:hyp-weak-form}
\end{align}
Note that constructing this weak-form from the PDE is not fully
rigorous. In fact, we must \emph{turn the process around} and forget
the PDE and instead use the weak-form to \emph{define} valid the
solutions to hyperbolic PDEs.

\begin{definition}[Weak-solution]
  A function $\mvec{Q}(x,t)$ is said to be a weak-solution if it
  satisfies the weak-form, \eqr{\ref{eq:hyp-weak-form}}, for all
  compact, smooth $\phi(x,t)$.
\end{definition}

A shock solution to a hyperbolic PDE is one that satisfies the
weak-form. Let $s$ the \emph{shock-speed} and the discontinuity exist
at some point $x$ and time $t$. Taking a small space-time interval
around $(x,t)$ we can show that the jump in
$\mvec{Q} = \mvec{Q}_R-\mvec{Q}_L$ and jump in flux
$\mvec{F} = \mvec{F}_R-\mvec{F}_L$ must be related by the
Rankine-Hugoniot jump condition
\begin{align}
  s(\mvec{Q}_R-\mvec{Q}_L) = \mvec{F}_R-\mvec{F}_L.
\end{align}

Plugging in the specific case for Burgers equation we get
\begin{align}
  s(u_R - u_L) = \frac{1}{2}( u_R^2 - u_L^2 )
\end{align}
or that
\begin{align}
  s = \frac{1}{2}( u_R + u_L ).
\end{align}
Hence, a shock front in Burgers equation will move with the
\emph{mean} of the characteristic speeds computed from the left/right
states.

\begin{figure}
  \setkeys{Gin}{width=0.95\linewidth,keepaspectratio}
  \incfig{burgers-rarefaction.pdf} 
  \caption{Diverging charactertics (right) leave behind a ``void'' in
    the $(x,t)$ plain, allowing us to fill it with many possible
    ways. This \emph{non-uniqueness} of solutions is solved by
    introducing \emph{entropy}, a scalar quantity that must be
    \emph{non-increasing} for all valid solutions of the PDE.}
  \label{fig:burgers-rare}
\end{figure}

Rather unfortunately, the weak-form of the solution does not uniquely
determine all solutions to hyperbolic PDEs. This is again seen in the
Burgers equations already, in the case when the characteristics
\emph{diverge}. See Fig.\thinspace\ref{fig:burgers-rare}. This shows
that when the characteristics diverge they leave behind a ``void'' in
the $(x,t)$ plain. We could fill this in many ways yet satisfy the
weak-form of the PDE. Hence, diverging characteristics lead to
non-uniqueness of solutions!

One way to fix this non-uniqueness is to introduce a \emph{entropy}
function. This is a scalar function, $s = s(\mvec{Q})$ that satisfies
the PDE
\begin{align}
  \pfrac{s}{t} + \pfrac{{f}_s}{x} \le 0
\end{align}
where ${f}_s = {f}_s\big(s(\mvec{Q})\big)$ is an \emph{entropy
  flux}. That is, the entropy is a \emph{non-increasing} function of
time. Imposing that entropy be \emph{decreasing} across any solution
we construct across diverging characteristics, picks out a unique
solution. This solution is called a \emph{rarefaction} wave, and for
Burgers equation is show on the left of
Fig.\thinspace\ref{fig:burgers-rare}.

\end{document}

