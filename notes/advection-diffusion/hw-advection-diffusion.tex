\documentclass[12pt]{article}

\usepackage{tikz}
\usetikzlibrary{decorations.markings}

%\usepackage{geometry}
%\geometry{paperwidth=170mm, paperheight=240mm, left=42pt, top=40pt, textwidth=280pt, marginparsep=20pt, marginparwidth=100pt, textheight=560pt, footskip=40pt}

%\documentclass[12pt]{amsart}
\usepackage[utf8]{inputenc}
\usepackage{times}
\usepackage{geometry}
\usepackage{tabularx}
\usepackage{amsmath}
\usepackage{amsfonts}
\usepackage{amsthm}
\usepackage{amssymb}
\usepackage{stmaryrd}
\usepackage{color}
\usepackage{sidecap}
\usepackage{physics}
\usepackage{graphicx}
\usepackage{color}
\usepackage{verse}
\usepackage{hyperref}
\usepackage{tensor}
\usepackage{marginnote}
\usepackage{wrapfig}
\usepackage[font={small}]{caption}

\newcommand{\gke}{\texttt{Gkeyll}}

\newcommand{\attrib}[1]{
\nopagebreak{\raggedleft\footnotesize #1\par}\vskip0.1in}
\renewcommand{\poemtitlefont}{\normalfont\large\itshape\centering}

\newtheorem{proposition}{Proposition}
\newtheorem{theorem}{Theorem}
\newtheorem{lemma}{Lemma}
\newtheorem{remark}{Remark}
\newtheorem{definition}{Definition}
\newtheorem{principle}{Principle}
\theoremstyle{definition}
\newtheorem{exmp}{Example}

\theoremstyle{definition}
\newtheorem{entry}{Entry}

\theoremstyle{definition}
\newtheorem{exer}{Exercise}
\newtheorem{prob}{Problem}

% auto-scaled figured
\newcommand{\incfig}{\centering\includegraphics}
\setkeys{Gin}{width=0.9\linewidth,keepaspectratio}

% Commonly used macros
\newcommand{\eqr}[1]{Eq.\thinspace(#1)}
\newcommand{\pfrac}[2]{\frac{\partial #1}{\partial #2}}
\newcommand{\pfracc}[2]{\frac{\partial^2 #1}{\partial #2^2}}
\newcommand{\pfraccc}[1]{\frac{\partial^2 }{\partial #1^2}}
\newcommand{\pfraca}[1]{\frac{\partial}{\partial #1}}
\newcommand{\pfracb}[2]{\partial #1/\partial #2}
\newcommand{\pfracbb}[2]{\partial^2 #1/\partial #2^2}
\newcommand{\spfrac}[2]{{\partial_{#1}} {#2}}
\newcommand{\mvec}[1]{\mathbf{#1}}
\newcommand{\bmvec}[1]{\breve{\mathbf{#1}}}
\newcommand{\gvec}[1]{\boldsymbol{#1}}
\newcommand{\script}[1]{\mathpzc{#1}}
\newcommand{\gcn}{\nabla}
\newcommand{\gcs}{\nabla_{\mvec{x}}}
\newcommand{\gvs}{\nabla_{\mvec{v}}}

\newcommand{\resetall}{\setcounter{entry}{0}\setcounter{equation}{0}\setcounter{footnote}{0}}

\newcommand{\cbas}[1]{\gvec{\sigma}_{#1}}
\newcommand{\xbas}{\gvec{\sigma}_{1}}
\newcommand{\ybas}{\gvec{\sigma}_{2}}
\newcommand{\zbas}{\gvec{\sigma}_{3}}

\newcommand{\basis}[1]{\mvec{e}_{#1}}
\newcommand{\bbasis}[1]{\breve{\mvec{e}}_{#1}}
\newcommand{\dbasis}[1]{\mvec{e}^{#1}}
\newcommand{\bdbasis}[1]{\breve{\mvec{e}}^{#1}}

\newcommand{\nbasis}[1]{\hat{\mvec{e}}_{#1}}
\newcommand{\ndbasis}[1]{\hat{\mvec{e}}^{#1}}

\newcommand{\gbasis}[2]{\mvec{#1}_{#2}}
\newcommand{\gdbasis}[2]{\mvec{#1}^{#2}}

\newcommand{\veps}{\gvec{\varepsilon}}
\newcommand{\bdum}{\breve{\_}}
\newcommand{\vecspace}{\mathcal{V}}
\newcommand{\tvecspace}{T_pM}
\newcommand{\vnorm}[1]{\lVert{#1}\rVert}

\newcommand{\gsel}[2]{\langle{#1}\rangle_{#2}}
\newcommand{\gzsel}[1]{\langle {#1} \rangle}

\newcommand{\uln}[1]{\underline{#1}}

\newcommand{\dprod}{\mathfrak{D}}

\newcommand{\dvol}{\thinspace d^3\mvec{x}}
\newcommand{\dsurf}{\thinspace ds}

% Make the items smaller
\newcommand{\cramplist}{
	\setlength{\itemsep}{0in}
	\setlength{\partopsep}{0in}
	\setlength{\topsep}{0in}}
\newcommand{\cramp}{\setlength{\parskip}{.5\parskip}}
\newcommand{\zapspace}{\topsep=0pt\partopsep=0pt\itemsep=0pt\parskip=0pt}

\newcounter{probnum}
\setcounter{probnum}{1}

%\marginpar{\raggedleft\footnotesize Margin note}

\title{Homework 1: Advection-Diffusion Equation, Finite-Difference Schemes}%
\author{AST560 2025}%
\date{Due: Thursday March 13th 2025}

\begin{document}
\maketitle

\section*{Problem \arabic{probnum}: Anisotropic Diffusion}
\stepcounter{probnum}

Consider the advection-diffusion equation, but generalize the
diffusion term to be \emph{anisotropic}
\begin{align}
  \pfrac{f}{t}
  +
  \gcn\cdot(\mvec{u} f)
  =
  \gcn\cdot(\mvec{D}\cdot \gcn f).
  \label{eq:adv-aniso-diff}
\end{align}
Here $\mvec{D}$ is the second-order \emph{diffusion
  tensor}. Anisotropic diffusion occurs in all magnetized plasmas: the
magnetic field provides a \emph{preferred} direction, along which the
heat flows much more easily than perpendicular to it. Assume that the
flow is incompressible. Repeat the proof for $L_2$ conservation but
for this system and derive the condition on the tensor $\mvec{D}$ for
the monotonic decay of the $L_2$ norm of the solution.

\section*{Problem \arabic{probnum}: $L_2$ Norm for Compressible Flow}
\stepcounter{probnum}

\begin{wrapfigure}[8]{L}{0.35\textwidth}
\incfig{hw1-ad-ic.png} 
\end{wrapfigure}
The proof for $L_2$ conservation requires that the flow be
incompressible. In this problem you will derive an explicit solution
to a \emph{compressible} flow problem, and show that $L_2$-norm is
indeed not conserved. Consider the 1D advection equation
\begin{align}
  \pfrac{f}{t} + \pfraca{x}(u f) = 0
\end{align}
where $u = u_L$ when $x<0$ and $u = u_R$ when $x\ge 0$. Assume both
$u_L$ and $u_R$ are positive. Consider the initial condition (a
rectangular pulse of height $H_L$ and width $W_L$) as shown in the
figure. Derive the exact solution at a later time $t > 0$ when the
pulse has completely moved to the right half of the domain. Using this
compute the $L_2$ norms of the initial and final solutions, showing
that the $L_2$ norm can increase \emph{or} decrease depending on the
ratio $u_L/u_R$.

\section*{Problem \arabic{probnum}: Upwind-Biased Scheme for Advection}
\stepcounter{probnum}

Consider the 1D advection equation
\begin{align}
\pfrac{f}{t} + \pfraca{x}(u f) = 0  
\end{align}
where $u > 0$ is a constant speed. Derive an \emph{upwind-biased}, 
semi-discrete scheme for this equation. This scheme should look like
\begin{align}
  \frac{df_0}{dt}
  +
  \frac{G^R - G^L}{\Delta x}
  =
  0.
\end{align}
where $G^R$ and $G^R$ are the numerical fluxes at the right/left
interfaces respectively. Use Fig.\thinspace[4] from the notes as a
guidance, and the third-order edge extrapolation formulas listed in
Section\thinspace 3.2

Use a Taylor series expansion to determine the order of convergence
for this scheme. Finally, derive the numerical dispersion relation
using Fourier analysis, and plot the real and imaginary parts of the
dispersion relation.

\section*{Problem \arabic{probnum}: Five-Point Stencil Scheme for
  Diffusion}
\stepcounter{probnum}

Consider the 1D diffusion equation
\begin{align}
\pfrac{f}{t} = \alpha \pfracc{f}{x}
\end{align}
where $\alpha$ is constant. We have already derived the semi-discrete
second-order, three-point stencil scheme for this as
\begin{align}
  \pfrac{f_h}{t} = \alpha \frac{f_R - 2 f_0 + f_L}{\Delta x^2}.
\end{align}
Derive a higher-order scheme using \emph{five} values
$f_{LL}, f_{L}, f_{0}, f_{R}, f_{RR}$ (where $LL$ and $RR$ are values
in cells to the left of $L$ and right of $R$ respectively.

Use a Taylor series expansion to determine the order of convergence
for this scheme. Finally, derive the numerical dispersion relation
using Fourier analysis, and plot it.

\section*{Problem \arabic{probnum}: Numerical Diffusion in First-Order
  Updwind Scheme}
\stepcounter{probnum}

Consider the semi-discrete, first-order upwind scheme for linear
advection:
\begin{align*}
  \frac{df_j}{dt} + \frac{f_j^n - f_{j-1}^n}{\Delta x} = 0.
\end{align*}
Though this scheme preserves positivity and monotonicity, it is very
diffusive. Rewrite this scheme as a combination of a
centered-difference scheme for the first derivative and a
centered-difference scheme for the second derivative. Hence,
explicitly derive the \emph{numerical diffusion} coefficient of the
first-order scheme.

\section*{Problem \arabic{probnum}: Flux form of Semi-Discrete Scheme,
  and Discrete $L_2$ Conservation}
\stepcounter{probnum}

We can write a generic semi-discrete scheme in the form
\begin{align}
  \frac{df_i}{dt} + \frac{G_{i+1/2}- G_{i-1/2} }{\Delta x} = 0
\end{align}
where $G_{i\pm 1/2}$ are the numerical fluxes. (See Section\thinspace
4 where we wrote the scheme for advection-diffusion equation in this
form). Then, the specific scheme we use (upwind, central, etc) all can
be written by an appropriate choice of numerical fluxes. Write down
the numerical fluxes (for example $G_{i-1/2}$) for (a) first-order
upwind scheme for advection equation, (b) second-order central scheme
for constant diffusion equation.

To understand how $L_2$ evolves for a generic semi-discrete scheme
written above, multiply by $f_i$ and sum over $i$, and use
index-shifting to rewrite the second term. Use this generic form and
the numerical fluxes for the second-order central scheme for constant
diffusion to show that the $L_2$ norm decays, as it should.

(As you showed in the previous problem that a first-order upwind
scheme can be written as a combination of central schemes for
advection and diffusion, this problem shows that the $L_2$ norm for
first-order upwind for advection decays monotonically).

\section*{Problem \arabic{probnum}: Stability of Implicit
  Time-Stepping Schemes and the Simple-Harmonic Oscillator}
\stepcounter{probnum}

Consider the simple first-order ODE
\begin{align}
  \dot{f} = (\lambda-i\omega) f
\end{align}
where $\lambda \le 0$. For this we will construct two discrete
schemes: the \emph{backward-Euler scheme}:
\begin{align}
  \frac{f^{n+1} - f^n}{\Delta t} 
  =
  (\lambda-i\omega) f^{n+1}
\end{align}
and the \emph{time-centered scheme} 
\begin{align}
  \frac{f^{n+1} - f^n}{\Delta t} 
  =
  (\lambda-i\omega) \frac{1}{2}(f^{n+1} + f^n).
\end{align}
Derive the stability properties of these schemes and show that they
are \emph{unconditionally stable}. What is the advantage of using the
time-centered scheme? (Hint: Consider the properties of $|f|^2$ in the
limit $\lambda = 0$).

Apply these implicit schemes to the harmonic oscillator
\begin{align}
  \frac{d^2z}{dt^2} = -\omega^2 z
\end{align}
by rewriting as a system of first-order ODEs
\begin{align}
  \frac{dz}{dt} = v; \quad \frac{dv}{dt} = -\omega^2 z.
\end{align}
Derive the exact conserved quadratic invariant for this system and
show that the time-centered scheme conserves this invariant exactly.

Introduce energy-angle coordinates:
\begin{align}
  \omega z = E\sin\theta; \quad v = E\cos\theta
\end{align}
As a challenge, derive an expression for the discrete phase-errors for
the time-centered scheme, that is, attempt to derive a Taylor series
expression for $\tan\theta^{n+1} = \tan(\theta^n + \omega\Delta t)$,
and see how many terms it matches, thus deriving the phase-error.

(The Boris algorithm used in particle-in-cell (PIC) codes is a
time-centered scheme for the $\mvec{v}\times\mvec{B}$ term, and has
structure and properties very similar to the one for the harmonic
oscillator. Standard PIC codes use time-centering for Maxwell
equations as well as for the particle push to allow stepping over the
cyclotron frequency, but must resolve the plasma-frequency, and of
course, the speed of light).

\end{document}